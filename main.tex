\documentclass{article}
\usepackage{dsfont}
\usepackage{amsmath} % to create a \cste operator 
\usepackage{cancel}
\usepackage{amsthm}
\usepackage{amssymb}
\usepackage{amsfonts} 
\usepackage{mathrsfs}
\usepackage{enumerate}
\usepackage{url}
\usepackage{pgf}
\usepackage{tikz}
\usepackage{cite}
\usepackage{graphicx}
\usepackage[colorlinks=true]{hyperref}
\usepackage[all]{xy}
\usepackage{amsmath,calligra,mathrsfs}
\DeclareMathOperator{\cste}{cste}
% operators
\DeclareMathOperator{\aut}{Aut}
\DeclareMathOperator{\End}{End}
\DeclareMathOperator{\Hom}{Hom}
  \DeclareMathOperator{\br}{Br}
  \DeclareMathOperator{\cl}{cl}
  \DeclareMathOperator{\cosp}{cosp}
  \DeclareMathOperator{\dv}{div}
  \DeclareMathOperator{\Div}{Div}
  \DeclareMathOperator{\ext}{Ext^{1}}
  \DeclareMathOperator{\Ext}{\mathscr{E}\text{\kern -2pt {\calligra\large xt}}\,\,} % sheaf ext
  \DeclareMathOperator{\gal}{Gal}
  \DeclareMathOperator{\gl}{GL}
  \DeclareMathOperator{\gr}{gr}
  \DeclareMathOperator{\h}{H}
  \DeclareMathOperator{\hh}{\mathsf{H}} % hypercohomology?
  \DeclareMathOperator{\shom}{\mathscr{H}\text{\kern -3pt {\calligra\large om}}\,} % sheaf hom
  \DeclareMathOperator{\im}{im}
  \DeclareMathOperator{\ob}{Ob}
  \DeclareMathOperator{\pgl}{PGL}
  \DeclareMathOperator{\pic}{Pic} 
  \DeclareMathOperator{\res}{R} % restriction of scalars
  \DeclareMathOperator{\rHom}{\mathrm{R}\vspace{-1pt}\mathscr{H}\text{\kern -3pt {\calligra\large om}}\,}
  \DeclareMathOperator{\sh}{Sh}
  \DeclareMathOperator{\spe}{sp} % specialisation
  \DeclareMathOperator{\spec}{Spec}
  \DeclareMathOperator{\swan}{Sw}
  \DeclareMathOperator{\tor}{Tor}
  \DeclareMathOperator{\Tor}{\mathscr{T}\text{\kern -4pt {\calligra\large or}}\,} % sheaf tor
  \DeclareMathOperator{\tr}{Tr}
 \DeclareMathOperator{\rg}{rg}
 \DeclareMathOperator{\degr}{deg}
%\DeclareMathOperator{\dim}{dim}
\DeclareMathOperator{\coh}{Coh(\mathbb{P}_{k}^{1})}
\DeclareMathOperator{\P1}{\mathbb{P}_{k}^{1}}
\DeclareMathOperator{\cok}{Coker}
\DeclareMathOperator{\sn}{\mathfrak{S}_n}
\DeclareMathOperator{\Fq}{\mathbb{F}_{q}}
\DeclareMathOperator{\hac}{H(\coh)}
\DeclareMathOperator{\hactx}{H(\coh_{\tor[x]})}
\DeclareMathOperator{\hact}{H(\coh_{\tor})}
\DeclareMathOperator{\hacl}{H(\coh_{l})}
\DeclareMathOperator{\alpv}{\alpha^{\vee}}
\DeclareMathOperator{\ad}{ad}
\DeclareMathOperator{\mult}{mult}
\DeclareMathOperator{\go}{\overset{o}{\mathfrak{g}}}
\DeclareMathOperator{\pio}{{\overset{o}{\Pi}}}
\DeclareMathOperator{\h0}{\overset{o}{\mathfrak{h}}}
\DeclareMathOperator{\w0}{\overset{o}{W}}
\DeclareMathOperator{\hod}{\overset{o}{\mathfrak{h}^{\ast}}}
\DeclareMathOperator{\Deltao}{\overset{o}{\Delta}}
\DeclareMathOperator{\sl2}{Sl_2}
% les bras et les kets
\newcommand{\bra}[1]{\langle\,#1\,|}
\newcommand{\ket}[1]{|\,#1\,\rangle}
\newcommand{\braket}[2]{\ensuremath{\langle\, #1 \mid  #2\, \rangle }}
\newcommand{\moy}[1]{\langle\,#1\,\rangle}
\newcommand{\vac}{\mathrm{vac}}
\newcommand{\zero}{\mathbb{0}}
%Prop, def, theo cadre maths:
\theoremstyle{definition}
\newtheorem{theo}{Theorem}[section]
\newtheorem{Prop}{Proposition}[section]
\newtheorem{ex}{Example}[section]
\newtheorem{cor}{Corollaire}[section]
\newtheorem{lemma}{Lemme}[section]
\newtheorem{preuve}{Preuve}[section]
\newtheorem{rem}{Remarque}[section]
\newtheorem{cdt}{Condition}[section]
\newtheorem{Def}{Definition}[section]
\author{Guillaume Corlouer}
\title{Integrable models in statistical physics, quantum groups and knot theory}
\begin{document}
\maketitle
\pagebreak
\tableofcontents
\pagebreak
\section{Introduction}
These notes are a very rough introduction to integrability in statistical physics, knot theory and quantum groups. We first present the concept of integrability in statistical physics through the 6 vertex model and explain that this concept is related to the Yang-Baxter equation. Solutions of this equations are called $R$ matrices and their commutation relations allow us to span a Yang-Baxter algebra that led mathematicians to define the notion of a quantum group. These quantum groups which are not quantum nor groups, are some kind of deformation of enveloping Lie algebras which bear the structure of a Hopf algebra. They have the important property to be noncommutative noncocommutative and self dual. We try to explore different interpretation of these quantum group : as 1 parameter deformation of the differential of regular functions over some Lie group or as 1 parameter formal series that are Lie algebra valued. We also take a small journey to the world of knots defining the Jones polynomial of a Link which is an algebraic invariant. We also explain how $R$ matrices give rise to representation of Artin's braid group. Finally we also look at tangles and the fact that coloring tangles can help us to define a quantum invariant of a knot through the notion of a Ribbon category. Interactions between statistical physics, quantum groups and knot theory are really rich (Topological quantum field theory, integrability, representation theory and other funny technology...) and these notes are just a very very tiny glimpse at the topic. Between examples and definitions there is sometimes blank space available so that the artist leaving in the reader's heart may draw knots links and tangles to visualize complicated and abstract equations written. The main reference for the statistical physics part is \cite{GomezRuizSierra} and for quantum group/knot theory part we  use \cite{Kassel} and \cite{KasRossTur}. Have fun ! 
\pagebreak
\section{Integrability in statistical physics}
The goal of physics is to understand systems of particles interacting with each others and to make predictions about their behavior. When the number of degrees of freedom is large, for example the number of particles in one $m^3$ of air, theoretical physicists use statistics by coarse graining the individual particles in order to analyse their collective behavior : statistical physics is a theory that allows to go from the microscopic to the macroscopic scale.\vspace{0.5cm} 

The time evolution of a physical system is governed by a linear operator called the Hamiltonian. Physically, this operator represents the energy of the system. A key function in statistical physics at thermodynamic (or thermal) equilibrium is the partition function which defines as follow : $$ Z=\tr_{\mathcal{H}}\left(\exp\left(-\frac{H}{kT}\right)\right)$$ where $H$ is a linear operator acting on the Hilbert space $\mathcal{H}$ of the states of the system, $T$ is the temperature and $k$ the Boltzmann constant, a fundamental constant that appears in statistical physics. This function is crucial to understand thermal systems because one can derive interesting observable quantities from it like the internal energy $U$ of the system : $$U=\langle E\rangle=\frac{-1}{\beta}\frac{\partial Z}{\partial\beta}$$
Here is a vague definition of an integrable system to give a first insight at the topic.
\begin{Def} A system is exactly solvable (or integrable) if one can diagonalise the Hamiltonian exactly and express the correlation functions of the model in terms of usual functions.
\end{Def} A 2 point correlation function of some physical system with physical observables  $\mathcal{O}_1$ and $\mathcal{O}_2 $  writes : 
$$\moy{\mathcal{O}_1\,\mathcal{O}_2} 
      = \frac{\tr_\mathcal{H} \left( \, \mathcal{O}_1\,\mathcal{O}_2 \, e^{-H /kT}\right)}
       {\tr_\mathcal{H} \left(e^{-H /kT}\right)}$$
In general the Hamiltonian decomposes into a sum of the Hamiltonian of the systems where the particles are all free and an interaction term :  $$H=H_{free}+H_{int}$$ This interaction term often present some non linearities and break all hopes to solve exactly a given model. So often to understand real life physics, one has to make approximation like mean field theory or use diagrammatic method like Feynman diagrams expansion to compute measurable quantities. Other models that do not need approximations to be solved are called exactly solvable models or integrable models... We will focus on a special model where computations are not too technical and a lot of important ideas of integrable models come into the game : the celebrated 6 vertex model.
\section{The 6 vertex model and the Yang-Baxter equation}
A vertex model in statistical physics is a lattice in which exactly two lines intersect at each vertex  of the lattice. The 6 vertex model is a square lattice of size $N\times N$. One color each edge by $-1$ or $1$ which stands for an unoccupied/occupied state or equivalently spin down/spin up. The constraint that the sum of the colored edges around the same vertex must be zero defines the 6 vertex model because this gives exactly 6 possible configurations around one vertex. We attach a statistical weight $W_{\alpha_{1}\beta_{1}}^{\beta_2\alpha_1}$ to each vertex of the lattice, $\alpha_{1},\beta_{1},\beta_2,\alpha_1$ coloring each edge. Let $\mathcal{H}_{V}$ be the Hilbert space of vertical states, the partition function  writes : $$Z=\tr_{\mathcal{H}_{V}}(t(u,v,w)^N)$$ where $t(u,v,w)$ is the transfert matrix of the model that depends on three parameters which are the three statistical weight of the model (the others being deduce from u,v,w by some symmetry). Using peridic boundary condition, the transfert matrix is such that : $$\langle\beta| t(u,v,w)|\alpha\rangle=\underset{\mu}{\sum}W_{\mu_{1}\alpha_{1}}^{\beta_1\mu_2}W_{\mu_{2}\alpha_{2}}^{\beta_2\mu_3}....W_{\mu_{N}\alpha_{N}}^{\beta_N\mu_1}$$ This quantity is the probability for a vertical state $$|\alpha\rangle=|\alpha_1\rangle\otimes...\otimes|\alpha_N\rangle$$ to be in a vertical state $$|\beta\rangle=|\beta_1\rangle\otimes...\otimes|\beta_N\rangle$$
To each horizontal edge one attach a finite dimensional vector $V_{a}$ space that we will call an \textit{auxiliary} space. To each vertical edge $i\in\{1,..,N\}$, one attach a vector space $V_{i}$. One can interpret the vertical direction as time and horizontal direction as space. The transfer matrix plays the role of a time evolution operator and is an endomorphism of $\mathcal{H}_{h}=V_{1}\otimes V_{2}\otimes...\otimes V_{N}$ where $\mathcal{H}_{h}$ is the space of horizontal edges. One may see a Boltzmann weight $ W_{\mu_{1}\alpha_{1}}^{\beta_1\mu_2}$ as an endomorphism $\mathcal{R}_{ai}$ of  $V_{a}\otimes V_{i}$ where $a$ is an index that parametrizes horizontal spaces and $i$ vertical spaces such that : $$W_{\mu_{i}\alpha_{i}}^{\beta_i\mu_{i+1}}=\mathcal{R}_{\mu_{i}\alpha_{i}}^{\mu_{i+1}\beta_{i}}$$ Thus one can rewrite the transfert matrix as follow : $$ t_{a}=\tr_{a}\left(\mathcal{R}_{aN}\mathcal{R}_{aN-1}...\mathcal{R}_{a1}\right)$$ Tracing out on the auxiliary space and multiplying  the $R$ operator over the vertical edges. Our goal is to diagonalize the transfer matrix since it is equivalent to diagonalize it or the hamiltonian (the vague reason for that is that the transfert matrix plays the role of a time evolution operator like the hamiltonian). We will look for a constraint that ensures the commutativity of any two transfer matrix. This constraint will be the so called Yang-Baxter equation. Let $t_{a}$ and $t'_{b}$ be any two transfer matrices. One derives : $$t_{a}t'_{b}=\tr_{a\times b}\left(\mathcal{R}_{aN}\mathcal{R}'_{bN}\mathcal{R}_{aN-1}\mathcal{R}'_{bN-1}...\mathcal{R}_{a1}\mathcal{R}_{b1}\right)$$ Thus using the cyclicity of the trace one sees that $t_{a}$ and $t'_{b}$ commut if and only if there exists an invertible linear operator $\mathcal{R}''_{ab}$ such that : $$\mathcal{R}''_{ab}\mathcal{R}_{ai}\mathcal{R}'_{ai}\mathcal{R}''^{-1}_{ab}=\mathcal{R}'_{bi}\mathcal{R}_{ai}$$ Relabelling one dervies the Yang-Baxter equation : $$\mathcal{R}_{12}\mathcal{R}'_{13}\mathcal{R}''_{23}=\mathcal{R}''_{23}\mathcal{R}'_{13}\mathcal{R}_{12}$$ where $\mathcal{R}_{12},\mathcal{R}'_{13}, \mathcal{R}''_{23}$ are acting on $V_{1}\otimes V_{2}$, $V_{1}\otimes V_{3}$ and $V_{2}\otimes V_{3}$ respectively. The vector spaces $V_{i}$ are two dimensional so the $R$ matrix is 4 by 4. Considering the 6 vertex model the $R$-matrix takes the following form : $$\mathcal{R}^{6v}(u,v,w)=
  \left( {\begin{array}{cccc}
   u & 0 & 0 & 0 \\
   0 & v & w & 0 \\
   0 & w & v & 0 \\
   0 & 0 & 0 & u \\
  \end{array} } \right) $$ 
\begin{Def} Let $V$ be a finite dimensional vector space. Let $R$ be a linear automorphism of $V\otimes V$. One says that $R$ is a $R$ matrix if it solves the Yang-Baxter equation : $$(R\otimes id_{V})(id_{V}\otimes R)(R\otimes id_{V})=(id_{V}\otimes R)(R\otimes id_{V})(id_{v}\otimes R)$$
\end{Def}
\begin{Def} A vertex model in statistical physics is called \textit{integrable} or exactly solvable if it carries a $R$-matrix.
\end{Def}
\begin{ex} The \textit{flip} $\tau_{V,V}\in\aut(V\otimes V)$ that switches two vectors is a $R$ matrix
\end{ex}
\begin{ex} The matrix $\mathcal{R}^{6v}(u,v,w)$ is a $R$ matrix
\end{ex} 

\section{The Yang-Baxter algebra} The Yang-Baxter equation and the invertibility of the matrix $\mathcal{R}^{6v}(u,v,w)$ actually allow us to write it as a matrix $\mathcal{R}(\lambda)$ that depends only on one complex parameter $\lambda$ such that :$$\mathcal{R}(\lambda):=\mathcal{R}^{6v}(u(\lambda),v(\lambda),w(\lambda))$$ The Yang Baxter equation now reads : $$(\mathcal{R}(\lambda)\otimes Id_2)(Id_2\otimes\mathcal{R}(\lambda+\mu))(\mathcal{R}(\mu)\otimes Id_2)=(Id_2\otimes\mathcal{R}(\mu))(\mathcal{R}(\lambda+\mu)\otimes Id_2)(Id_2\otimes\mathcal{R}(\lambda))$$ One defines a crucial object called the monodromy matrix  $T(\lambda)$ : $$T(\lambda):=\mathcal{R}_{aN}\mathcal{R}_{aN-1}...\mathcal{R}_{a1}$$
Thus the trace of the monodromy matrix over the auxiliary space gives the transfer matrix. Hence the monodromy matrix is a $4\times 4$ matrix of operators A,B,C,D that depends on the weights of the lattice : \begin{equation}
T(\lambda)=\begin{pmatrix}
A(\lambda) & B(\lambda)\\
C(\lambda) & D(\lambda) \\
\end{pmatrix}
\end{equation}
Using the previous Yang-Baxter equation repetitively one may check that the monodromy verifies the important relation : \begin{equation}\label{YBmono}
R(\lambda-\mu)(T(\lambda)\otimes Id)(Id\otimes T(\mu))=(T(\mu)\otimes Id)(T(\lambda)\otimes Id) R(\lambda-\mu)
\end{equation} This gives rise to an algebra spanned by the operators A,B,C,D that satisfy some commutation relation like some of the following (none exhaustive list ) : \begin{align}
&[B(\lambda),B(\mu)]=0\qquad [C(\lambda),C(\mu)]=0\\
&[A(\lambda),A(\mu)]=0\qquad [D(\lambda),D(\mu)]=0\\
&A(\lambda)B(\mu)=w(\lambda,\mu)A(\mu)B(\lambda)+v(\lambda,\mu)B(\mu)A(\lambda)\\
&D(\mu)B(\lambda)=v(\lambda,\mu)B(\lambda)D(\mu)+w(\lambda,\mu)D(\lambda)B(\mu)\\
&C(\lambda)D(\mu)=v(\lambda,\mu)C(\mu)D(\lambda)+w(\lambda,\mu)D(\mu)C(\lambda)\\
&C(\mu)A(\lambda)=v(\lambda,\mu)A(\lambda)C(\mu)+w(\lambda,\mu)C(\lambda)A(\mu)\\
&[C(\lambda),B(\mu)]=w(\lambda,\mu)v(\lambda,\mu)^{-1}[A(\lambda)D(\mu)-A(\mu)D(\lambda)]
\end{align} To diagonalize the transfer matrix is equivalent to diagonalize A and D simultaneously. To do this one use the operators B and D as creation and annihilation operators and the commutation relations of the previous algebra. We'll not go into these technical details here (which is known as the algebraic Bethe Ansatz)  but refer the interested reader to \cite{GomezRuizSierra}(section 2).
\begin{Def} A Yang-Baxter algebra is a couple $(\mathcal{R}, T)$ such that the quadratic relation \ref{YBmono} is true.
\end{Def} The link with quantum groups is here : it turns out that in the limit where the spectral parameter $\lambda\to\infty$ the Yang-Baxter Algebra identifies with some quantum group. 
\section{Hopf algebra}
Let $k$ be a field and $A$ a $k$-module.\begin{Def} An algebra is a triple $(A,\eta,\mu)$  with unity $\eta : k\to A$ and a product $\mu: A\otimes A\to A$ such that: 
\begin{align*} 
\mbox{associativity}\quad\mu(id\otimes\mu) & =\mu(\mu\otimes id) \\
\mbox{unitality}\quad\mu(id\otimes\eta) & =\mu(\eta\otimes id)=id_A \\
\mu^{op} & =\mu\tau_{A,A}
\end{align*} 
If $\mu=\mu^{op}$, one says that the algebra is commutative.
\end{Def}
\begin{Def} A coalgebra is a triple $(A,\epsilon,\Delta)$ with counity $\epsilon : A\to k $ and \textit{coproduct} $\Delta : A\to A\otimes A $ such that : \begin{align*} 
\mbox{coassociativity}\quad (id\otimes\Delta)\Delta & =\Delta(\Delta\otimes id) \\
\mbox{counitality}\quad (id\otimes\epsilon)\Delta & =(\epsilon\otimes id)\Delta=id_{A} \\
\Delta^{op} & =\tau_{A,A}\Delta
\end{align*}
If $\Delta^{op}=\Delta$ one says that the coalgebra is \textit{cocommutative}.
\end{Def}
\begin{Def} A \textit{bialgebra} is a tuple $(A,\mu,\eta,\Delta,\epsilon)$ such that $(A,\mu,\eta)$ is an algebra and  $(A,\Delta,\epsilon)$ is a coalgebra.
\end{Def}
\begin{ex}
$A^{op}=(A,\mu^{op},\eta,\Delta,\epsilon)$ , $A^{cop} (A,\mu,\eta,\Delta^{op},\epsilon)$ and $A^{op,cop}=(A,\mu^{op},\eta,\Delta^{op},\epsilon)$ are all bialgebras
\end{ex}
\begin{ex} The dual of a bialgebra is a bialgebra (switch product and coproduct with using duality functor)
\end{ex}
\begin{ex} The Yang-Baxter algebra of the 6 vertex model studied in the last section is a bialgebra where the coproduct $\Delta$ is given by : $$\Delta(T_{j,i}(\lambda))=\sum_{k}T_{ki}(\lambda)\otimes T_{jk}(\lambda)$$
\end{ex}
\begin{Def} A Lie algebra $\mathfrak{g}$ is an algebra with a product $[\quad,\quad]$ anticommutative and satisfying the Jacobi identity.
\end{Def}
\begin{ex} The enveloping algebra $U(\mathfrak{g})$  a Lie algebra $\mathfrak{g}$ (which one can see as the tensor algebra of the Lie algebra quotiented by the commutation relations of the element of $\mathfrak{g}$) is a bialgebra with coproduct $\Delta$ such that : $$\forall x\in U(\mathfrak{g}),\quad\Delta(x)=1\otimes x+ x\otimes 1\quad\mbox{and}\quad \epsilon(x)=0$$
So the enveloping algebra of a Lie algebra is a cocommutative bialgebra.
\end{ex}
\begin{Def} One can endow the vector space $\End(A)$ of a bialgebra $A$ with a \textit{convolution} product denoted $\star$ such that for all $f,g\in\End(A):$, $$f\star g=\mu(f\otimes g)\Delta $$
\end{Def}
\begin{Def} A Hopf algebra is a bialgebra in which the identity of $A$ as a two sided inverse $S$ for the convolution product. This inverse is called an \textit{antipode} and verifies : $$S\star id_A= id_A\star S=\eta\epsilon$$
A morphism of a Hopf algebra is a morphism of the underlying bialgebra that commutes with the antipode.
\end{Def}
\begin{ex}
Consider $U(\mathfrak{g})$ as the bialgebra defined previously. The $k$ linear map $S$ such that : $S(1)=1$ and $$\forall x_1,x_2,..x_n\in\mathfrak{g},\quad S(x_1x_2...x_n)=(-1)^nx_n..x_2x_1$$ is an antipode an thus allows us to construct a structure of Hopf algebra on $U(\mathfrak{g})$.
\end{ex}
\section{Quantum groups}
We are now ready to understand the notion of a quantum group. The name can be misleading because a quantum group is not a group nor quantum! Let us consider the Lie group $Sl_2$ of matrices of unit determinant and whose associated lie algebra $sl_2$ is the algebra of traceless matrices, over the complex numbers. Recall that $sl_2$ can be presented as generators $h,e,f$ with commutation relations such that : \begin{align*}
[h,e]&=2e\\
[h,f]&=-2f\\
[e,f]&=h
\end{align*}
The cartan subalgebra is generated by $h$ and $e$ is the generator of a borel(positive) subalgebra. Roughly a quantum group will be interpret as a deformation of the lie algebra, i.e a family of algebra $U_q(sl_2)$ paramterized by a parameter $q$ such that when one specialized $q=1$ one recovers the classical enveloping  Lie algebra $U(sl_2)$. More precisely we will interpret it as a $q$-deformation of the algebra of the  derivations of regular functions over the Lie group $Sl_2$. Let us recall that one can construct structure of Hopf algebra on the algebra of regular functions $$\mathbb{C}[Sl_2]=\mathbb{C}[x_{11},x_{12},x_{21},x_{22},\det^{-1}]$$. The coproduct  being given by : $$\Delta(x_{ij})=\sum_k x_{ik}\otimes x_{kj} $$, the counit by $\epsilon(x_{ij})=\delta_{ij}$ and the antipode by the inversion of matrix. We introduce the notion of \textit{Hopf pairing} which allows us do define a notion of duality between Hopf algebras.\begin{Def}
Let $A$ and $B$ be two Hopf algebras. A Hopf pairing is a bilinear form $(-,-) : A\otimes B\to k$ such that : \begin{align*}
(a,bb')&=(\Delta_A(a),b\otimes b')\\
(aa',b)&=(a\otimes a',\Delta_B(b))\\
(a,1_B)&=\epsilon(a)\\
(Sa,b)&=(a,S^{-1}b)\\
\end{align*}
If the bilinear form is non degenerate one says that the pairing is \textit{perfect}
\begin{ex}\label{classical duality} One may check that the Hopf algebras $\mathbb{C}[Sl_2]$ and $U(sl_2)$ are dual, with the pairing satisfying (for example) : $$\forall u,v\in U(sl_2),\quad \langle uv,x_{11}\rangle = \langle u,x_{11}\rangle\langle v,x_{11}\rangle+\langle u,x_{12}\rangle\langle v,x_{21}\rangle$$
\end{ex}
\end{Def}
It is important to note that the Hopf algebra $\mathbb{C}[Sl_2]$ which is commutative but not cocommutative and $U(sl_2)$ which is not commutative but cocommmutative are dual. The quantum group $U_q(sl_2)$ will bear the property to be a non cocommutative, not commutative, self dual Hopf algebra. By analogy with the previous classical case this self duality property allows us to think of the quantum group $U_q(sl_2)$ as a non commutative algebra of regular functions over some non commutative space and whose limit is the algebra of function over some Lie group. Whence the name "quantum group".
\subsection{The quantum plane}
\begin{Def} One call the algebra of function over the complex plane quotiented by the relation $xy=qyx$ for sme indeterminate $q$ the quantum plane. In other words : $$\mathbb{C}_q\langle x,y\rangle:=\mathbb{C}[x,y]/(xy-qyx)$$
There is a coaction of the Hopf algebra $\mathbb{C}[Sl_2]$ on the plane $\mathbb{C}[x,y]$ derived from matrix multiplication on a vector $(x,y)\in\mathbb{C}^2$ In order to define a deformation of the Hopf algebra $\mathbb{C}[Sl_2]$ we will look for a condition on the entries of a 2 by 2 matrix  that ensures that the relation $xy=qyx$ in the quantum plane is satisfied. \begin{Prop} Let $M=
  \left( {\begin{array}{cc}
   a & b \\
   c & d \\
  \end{array} } \right)$ and $X=(x,y)$, $X'=(x',y')$, $X''=(x'',y'')$ be three points of the quantum plane. The conditions : $$MX=X'\quad M^tX=X''$$ give the following commutation relations :\begin{align*} ba&=qba\qquad db=qbd \\
  ca&=qac\qquad dc=qcd\\
  bc&=cb\qquad ad-da=(q^{-1}-q)bc\\
\end{align*}
We will note $I_q$ the ideal spanned by these relations. 
\end{Prop}
\end{Def}
\begin{Def} Let M be a 2 by 2 matrix. One defines the quantum determinant by $$\det_q=(M)=da-qbc$$ 
\end{Def}
\begin{Def} The quantum algebra of $Sl_2$ writes : $$Sl_q(2):=\mathbb{C}_q[Sl_2]:=\mathbb{C}[Sl_2]/(I_q,\det_q-1)$$
\end{Def}
\begin{Prop}There exists a Hopf algebra structure on $Sl_q(2)$ given by : \begin{align*}
\Delta(M)&=M\otimes M\\
\epsilon(M)&=Id_2\\
S(M)&=\left( {\begin{array}{cc}
   d & -qb \\
   -q^{-1}c & a \\
  \end{array} } \right)
\end{align*}
\subsection{Quantum differentials}
Recall that $sl_2$ is the algebra of differential of $\mathbb{C}[Sl_2]$ so, dually one may consider the action of the $sl_2$ on the complex plane through the identification \begin{align*}
&eP=x\frac{\partial}{\partial y} P \\
&fP=\frac{\partial}{\partial x} P\cdot y \\
&hP=x\frac{\partial}{\partial x}P-\frac{\partial}{\partial y}P\cdot y 
\end{align*}
With $P\in\mathbb{C}[x,y]$. One defines the $q$ derivative that are \textit{quantum} analog of the previous classical differentials :
\begin{Def} Let $m,n\in\mathbb{N}$ A $q$-derivative  of $\mathbb{C}_q\langle x,y\rangle$ satisfies: $$\frac{\partial_q}{\partial x}(x^my^n)=[m]_qx^{m-1}y^{n}, \quad \frac{\partial_q}{\partial y}(x^my^n)=[n]_qx^{m}y^{n-1}$$ \end{Def}
Where $[n]_q=\frac{q^n-q^{-n}}{q-q^{-1}}$ is the so called $q$-number.
One defines the generators $E,F,K$ of the quantum groups $U_q(sl_2)$ analog to the corresponding classical generators of $U(sl_2)$ $e,f,h$ by replacing the classical derivative by their quantum analog.
\subsection{Deforming the enveloping Lie algebra and self duality}
\begin{Def} Let $U_q(sl_2)$ be the algebra over $\mathbb{C}(q)$ spanned by $E,F,K$ defined by the following relations : \begin{align*}
&KEK^{-1}=qE\\
&KFK^{-1}=q^{-1}F\\
&[E,F]=\frac{K-K^{-1}}{q^{1/2}-q^{-1/2}}\\
\end{align*} One notes that $K$ plays the role of the generator of a kind of "quantum Cartan subalgebra" of $U_q(sl_2)$
\end{Def}
\end{Prop}
Given two elements $P,Q$ in the quantum plane one shows, using the $q$ derivative that : \begin{align}
&K(PQ)=K(P)K(Q)\\
&E(PQ)=PE(Q)+E(P)K'(Q)\\
&F(PQ)=K^{-1}(P)F(Q)+F(P)Q\\
\end{align}
This allows us to endow the quantum group with a structure of a Hopf algebra\cite{Kassel} (VII p150)
\begin{theo} The coproduct $\Delta$ such that :\begin{align*}
&\Delta(E)=1\otimes E+E\otimes K\\
&\Delta(F)=K^{-1}\otimes F+F\otimes 1\\
&\Delta(K)=K\otimes K\\
\end{align*}
The counit $\epsilon$ such that :\begin{align*}
&\epsilon(E)=\epsilon(F)=0\\
&\epsilon(K)=\epsilon(K^{-1})=1\\
\end{align*}
And the antipode $S$ such that : \begin{align*}
&S(E)=-EK^{-1}\\
&S(F)=-KF\\
&S(K)=K^{-1}\\
\end{align*}
endows the quantum group $U_q(sl_2)$ with the structure of Hopf algebra.
\end{theo}
A key property of quantum groups is the fact that they are self dual Hopf algebras : \begin{Prop}Let $U_q(\mathfrak{b}^+)$ the Hopf algebra spanned by $E$ and $K$. This algebra is self dual  via the following Hopf non degenerated Hopf pairing: $$\langle-,-\rangle : U_q(\mathfrak{b}^+)\otimes U_q(\mathfrak{b}^+)\to C(q)$$ such that \begin{align*}
&\langle E,E\rangle = 1\qquad \langle E, K\rangle =\langle K,E\rangle =0 \\
&\langle K,K\rangle = q
\end{align*}
\end{Prop}
According to \cite{Kassel} (theorem VII 4.4) there exist a duality between $U_q(sl_2)$ and $Sl_q(2)$ which is the quantum version of the classical duality \ref{classical duality} (when q=1). Thus the quantum group is really a non commutative, non cocommutative, self dual Hopf algebras. So one can interpret it in two different ways : \begin{itemize}
\item A $q$-deformation of a Hopf algebra associated to some enveloping Lie algebra through the $q$-deformation of the action of the algebra of differential of regular functions over some Lie group
\item A $q$-deformation the Hopf algebra of regular functions over some Lie group through its co-action on a noncommutative space
\end{itemize} 
We focused on the case $sl_2$, but we define quantum groups in general analogously : \begin{Def} Let $\mathfrak{g}$ be a Lie algebra with Cartan matrix $A=(a_{ij})_{0\leqslant i,j\leqslant l}$. The quantum Lie algebra $U_q(\mathfrak{g})$ is a $\mathbb{C}(q)$-algebra spanned by the family $\{K_i,E_i,F_i\vert i=0,..l\}$ with relations : 
\begin{align*}
&K_iK_j=K_jK_i\\
&K_iE_jK_i^{-1}=q^{a_{ij}/2}E_j\\
&K_iF_jK_i=q^{-a_{ij}/2}F_j\\
&[E_i,F_j]=\delta_{ij}\frac{K_i-K_j^{-1}}{q^{1/2}-q^{-1/2}}\\
&\underset{k=0}{\overset{1-a_{ij}}\sum}(-1)^{k}\left[\overset{1-a_{ij}}{k}\right] E_i^lE_jE_i^{1-a_{ij}-l}=0\\
&\underset{k=0}{\overset{1-a_{ij}}\sum}(-1)^{k}\left[\overset{1-a_{ij}}{k}\right] F_i^lF_jF_i^{1-a_{ij}-l}=0
\end{align*}
The two last relations are the quantum analog of the Serre's relations.
\end{Def}
\section{Knot theory}

\subsection{Setting goal}
\begin{Def} A \textit{link} is a finite collection of disjoint circles smoothly embedded in $\mathbb{R}^3$
\end{Def}
\begin{Def}
A \textit{knot} is a link with one connected component.
\end{Def}
\begin{Def} An \textit{isotopy} of a link is a smooth deformation into the class of links that does not create intersections nor self intersections.
\end{Def}
The goal of knot theory is to classify knots up to isotopy.
\subsection{Reidemeister moves} One conviniently represents links projected on the plane.
\begin{Def}A \textit{framing} of a projected link consists in choosing an orientation of the link and a sign $+1$ or $-1$ at each over/undercrossing \end{Def}
\begin{Def}
The \textit{framing number} of a link is the sum of the sign assigned at each crossing of a link
\end{Def}
We presents three elementary moves on links called the \textit{Reidemeister} moves :\pagebreak
\begin{theo} Two link diagrams are isotopic framed links if and only if they may be related by an isotopy and a finite sequence of Reidemeister moves.
\end{theo}
\begin{Def} Let $K$ and $K'$ be two oriented knots. Assign number $+1$ or $-1$ to each over/undercrossing of the link diagram of $K\cup K'$ where $K$ and $K'$ meet. The sum of the numbers is the \textit{linking number} denoted $lk(K,K')$ and is preserved under Reidemeister moves.
\end{Def}

We will define our first algebraic invariant of Links namely the Jones polynomial.
\subsection{Skein classes}
\begin{Def} Let $a$ be some complex number. Let $E(a)$ be the complex vector space spanned by all linked diagrams quotiented by : \begin{itemize}
\item Isotopy relation
\item the relation $D\cup O=-(a^2+a^{-2})D$ where $O$ is a simple circle in the complementary of $D$
\item the Kauffmann's skein relation that one obtained by smoothing the crossing of a diagram, there are two admissible resulting diagrams  given a choice of clockwise or anticlockwise smoothing of the initial crossing :  \vspace{10cm}
\end{itemize}
For a diagram $D$ we will denote $\langle D\rangle $ its skein class. The empty link diagram will be denoted $\langle \emptyset\rangle$.
\end{Def}
Applying the Kauffmann's skein relation iteratively to some diagram  $D$ and $\emptyset\cup O=-(a^2+a^{-2})\emptyset$  one obtain the following fact :
\begin{Prop}\label{Edim1}
The vector space $E(a)$ is one dimensional and $$E(a)=Vect(\langle\emptyset\rangle)$$ 
\end{Prop}
It's also funny to check the following : 
\begin{Prop}\label{skeininv}
The skein class of any link diagram is invariant under Reidemeister moves
\end{Prop}
\vspace{10cm}
\subsection{The bracket polynomial} Let $a$ be a complex number such that $a^2+a^{-2}$ is non zero. According to Kauffman recursive relation and \ref{Edim1} the skein class of any link diagram $D$ is a Laurent polynomial in $a$ divisible by $a^2+a^{-2}$. Furthermore according to \ref{skeininv} it is isotopy invariant. We thus define the following polynomial: 
\begin{Def} The bracket polynomial denoted $\langle L\rangle (a)$ is such that : $$\langle L\rangle (a):=-(a^2+a^{-2})^{-1}\langle D\rangle(a)$$
\end{Def}
\begin{ex} The bracket polynomial of the Hopf link is $$\langle L\rangle (a)=-a^4-a^{-4}$$ \vspace{5 cm}
\end{ex}
\subsection{The Jones polynomial}\begin{Def}
Let us consider a framed link $L$. Let $w(D)$ be the sum of the sign  of each crossing of the link diagram $D$ of $L$. Let $D$ be the number of crossing points of $D$. Let $q$ be a complex number. The Jones polynomial $V_{L}(q)$ is such that $$V_{L}(q):=(-1)^{|D|+1} q^{3w(D)/2}\frac{\langle D\rangle(q^{-1/2})}{q+q^{-1}}$$ 
\end{Def}
The Jones polynomial is isotopy invariant since the bracket polynomial is. Furthermore it is one on the trivial knot. It also bear the interesting following property :
\begin{Prop} Let $L_+$, $L_{-}$,$L_0$ be a Conway triple. Thus : $$q^{-2}V(L_{+})-q^{-2}V(L_{-})=(q-q^{-1})V(L_0)$$
\end{Prop}
\vspace{10cm}
\section{Tangles} Tangles are a bit more general than links. They are links that contain arcs with prescribed end points.
\begin{Def} Let $k,l$ be natural integers. A \textit{tangle} with $k$ inputs and $l$ outputs, or a $(k,l)$-tangle is a finite union of disjoint smoothly embedded arcs and circles in $\mathbb{R}^{2}\times [0,1]$ the end point of the arcs being $(1,0,0),(2,0,0)...,(k,0,0)$ and $(1,0,1),(2,0,1)...,(l,0,1)$.
\end{Def}
\begin{ex}
\vspace{10cm}

\end{ex}

We can define tangle diagram in the projection of the plane, framed tangled diagram, isotopy classes, skein classes and skein module in the same fashion than we have done for links. The goal of tangle theory is to classify tangles up to isotopy.
\subsection{The category of tangles} We define the category of tangles and the notion of a braiding in this category. Object of this category are natural integers $0,1,2,..$ and morphisms are arrows $k\to l $ that are isotopy classes of framed $(k,l)$-tangle. The composition $f\circ g$ being defined by attaching $f$ on the top of $g$ and compressing the resulting diagram into $\mathbb{R}\times [0,1]$. It's also worth to note that we have a tensor product functor such that for an object $k$ and $l$, $k\otimes l$ results in a $k+l$ and tensor product of morphism is juxtaposition of frame tangle diagrams without creating new intersections.
\begin{ex}
\vspace{10cm}
\end{ex}
\begin{Def} A braiding in the category of tangles is a morphism $c_{k,l}:k\otimes l \to l\otimes k$ that places a bunch of $k$ arcs above $l$ arcs resulting in a tangle diagram with $kl$  crossing points.
\end{Def}
\subsection{Braids}
\begin{Def}
A braid  is a particular tangle  that contains only arcs (or strands) i.e it has no circles
\end{Def} 
Similarly we have the notion of braid diagrams, framed braids, isotopy classes... 
\begin{Def} Let $n$ be an positive integer greater than 2. The braid group $B_n$ with n strands is an infinite group defined by $n-1$ generators $\sigma_1,..\sigma_{n-1}$ with relations : \begin{align*}
&\sigma_i\sigma_j=\sigma_j\sigma_i\quad\forall|i-j|>1\\
&\sigma_{i+1}\sigma_i\sigma_{i+1}=\sigma_{i}\sigma_{i+1}\sigma_{i}\quad\forall 1\leq i,j\leq n-1 \\
\end{align*}
\end{Def}
One notices at once that the symmetric group $S_n$ spanned by permutations $\sigma_i=(i,i+1)$ is the braid group quotiented by the relation $\sigma_i^{2}=1$. Let $V$ be a finite vector space and $R\in\aut(V\otimes V)$ a $R$ matrix. Let $R_i\in\aut(V^{\otimes n })$ such that $$R_i=id_{V^{i-1}}\otimes R\otimes id_{V^{\otimes n-i-1}}$$
The automorphisms $R_i$ satisfy the braid relations so one deduces the following fact : \begin{Prop} Let $R\in\aut(V\otimes V)$ be a solution of the Yang-Baxter equation. Then for any $n>0$ there exists a unique group homomorphism $\rho_n^{R}:B_n\to \aut(V^{\otimes n})$ such that $\rho_n^{R}(\sigma_i)=R_i$
\end{Prop}
Thus $R$ matrices give rise to representations of the braid group
\section{Braided categories}
\subsection{Monoidal category}
One key difference with commutative algebra is the fact that the canonical isomorphism of $A$-module $V\otimes W\simeq W\otimes V$ induced by the flip is not linear in general. Indeed given two $A$ modules $V$ and $W$  and a bialgebra $A$ one defines a structure of $A$-module over $V\otimes W$ by : $$a.(v\otimes w)=\Delta(a)(v\otimes w)=\sum_{i}a_{i}v\otimes a'_{i} w$$ Unless the bialgebra is cocomutative $a.(v\otimes w)\neq a.(w\otimes w) $ since $$a.(w\otimes v)=\Delta(a)\tau_{V,W}(v\otimes w)=\Delta^{op}(a)(v\otimes w)$$ We will see that this non commutativity of $A$ modules, due to the non cocommutativity of the coproduct will be measured in some sense by a universal $R$ matrix and will give rise to the notion of a braided bialgebra. For the time being, let us define the notion of \textit{strict monoidal category}:
\begin{Def} Let $\mathcal{C}$ be a category and $\otimes$ be a functor from $\mathcal{C}\times\mathcal{C}\to \mathcal{C}$. A \textit{strict monoidal category} with unit object 1  is a category such that for all object $V$,$W$,$U$ and all morphisms $f,g,h$ in $\mathcal{C}$ one has : 
\begin{align} 
(U\otimes V)\otimes W & =  U\otimes(V\otimes W) \\
(f\otimes g)\otimes h & = f\otimes(g\otimes h)\\
V\otimes 1 &  = 1\otimes V=V\\
f\otimes id_{1} & = id_{1}\otimes f=f 
\end{align}
%So the idea of a strict monoidal category is that the tensor product functor is associative for all object and morphisms of $\mathcal{C}$.
%We know that in commutative algebra, for example in the category of module over some commutative ring, there is a canonical isomorphism such that : $M\otimes N\simeq N\otimes M $. In the category that we are going to study, like category of tangles, or representation of quantum groups this is not the case anymore and one has to take into account the non-commutativity of the objects.
\end{Def}
\subsection{Braided bialgebra} We will build an isomorphism of object of some monoidal category $\mathcal{C}$ that will measure in some sense the noncommutativity of coproduct, namely the \textit{braiding}.
\begin{Def} Let $V$ and $W$ be two objects of $\mathcal{C}$. A commutativity constraint $R_{V,W}:V\otimes W\to W\otimes V$ is such that for all morphism $f\otimes g : V\otimes W\to V'\otimes W'$ one has $$R_{V',W'}(f\otimes g)=(g\otimes f)R_{V,W} $$
\end{Def}
\begin{Def} A \textit{braiding} is a commutativity constraint such that \begin{align*}
&R_{U\otimes V,W}=(R_{U,V}\otimes Id_W)(Id_U\otimes R_{V,W})\\
&R_{U,V\otimes W}=(Id_V\otimes R_{U,W})(R_{U,V}\otimes Id_W)\\
\end{align*}
\end{Def}
\begin{Def}
A \textit{braided monoidal category} is a monoidal category with a braiding.
\end{Def}
A crucial fact that one may check using tangle diagrams is that a braiding in a braided monoidal category solves the
Yang-Baxter equation.\\\\
The following theorem characterizes the categories of $A$ modules that are braided :
\begin{theo} Let $A$ be a bialgebra. The category of $A$ module is braided if and only if there exists an invertible elements $R\in A\otimes A$ such that : \begin{align*} 
&\Delta^{op}(x)=R\Delta(x) R^{-1}\\
&(\Delta\otimes id_A)(R)=R_{12}R_{13}\\
&(id_A\otimes\Delta)(R)=R_{13}R_{12}\\
\end{align*}
where $R_{12}=R\otimes id_A$, $R_{23}=id_A\otimes R$ and $R_{13}=(\tau_{A,A}\otimes id_A)(id_A\otimes R)$
The element $R$ is called a \textit{universal} $R$ matrix in the sense that it produces solutions of the Yang-Baxter equations. Intuitively the universal $R$ matrix measures the non cocommutativity of the coproduct.\vspace{0.5cm}
A braided bialgebra $A$ with universal $R$ matrix is a \textit{braided bialgebra}. If $A$ has an antipode, we say that $A$ is a \textit{braided  Hopf algebra}.
\end{theo}
\begin{theo}\cite{KasRossTur}(p55) The category of tangle with braiding $R_{k,l}$ defined previously is a braided monoidal category
\end{theo}
\section{Ribbon category and quantum invariant}
\subsection{Ribbon category}
\begin{Def} Let $\mathcal{C}$ be a monoidal category. Let $V$ be an object of $\mathcal{C}$ to which one associates an object $V^{*}$. Suppose we have morphisms : $$b_V= 1\to V\otimes V^{*}\quad d_V : V\otimes V^{*}\to 1 $$ One says that $(V,V^{*})$ is a duality in $\mathcal{C}$ if :
\begin{align*}
&(id_V\otimes d_V)(b_V\otimes id_V)=id_V\\
&(d_V\otimes id_{V^{*}})(id_{V^{*}}\otimes b_V)=id_{V^{*}}\\
\end{align*}
\begin{ex} The standard evaluation and coevaluation in the category of modules realizes a duality
\end{ex}
\end{Def}
\begin{Def} Let $\mathcal{C}$ be a monoidal category. Let $R_{V,W}$ be a braiding. A twist $\theta$ is a natural family of isomorphism $$\theta=\left\lbrace\theta_V : V\to V \right\rbrace $$ such that for any two object $V$ and $W$ of $\mathcal{C}$ one has :
$$\theta_{V\otimes W}=R_{W,V}R_{V,W}(\theta_V\otimes\theta_W)$$ Naturality here means that $\theta$ commutes with any morphism.
\end{Def}
\begin{Def}
A ribbon category is a monoidal category equipped with braiding duality and twist with the following compatibility : 
$$(\theta_V\otimes id_{V^{*}})b_V=(id_V\otimes\theta_{V^{*}})b_V$$
\end{Def}

\begin{ex} The category of tangles is a Ribbon category, whith $$\theta_k : k\to k\quad b_k : 0\to 2k\quad d_k : 2k\to 0 $$ \vspace{10cm}
\end{ex}
The following theorem is enlightening because it replaces equations with drawing tangle diagrams and we love drawing:
\begin{theo}\cite{KasRossTurross} (2.2 p67) Given a Ribbon category $\mathcal{C}$ with braiding we can always color the objects and morphisms of the category of tangles with the objects of  $\mathcal{C}$. This give rise to a functor from the category of framed color tangles to $\mathcal{C}$ that preserves the tensor product.
\end{theo}\vspace{10cm}
\subsection{Quantum invariant}
An other way to construct a quantum group from a Lie algebra $\mathfrak{g}$ is to quantize it into a $\mathbb{C}[[h]]$-module of $\mathfrak{g}$ valued formal series such that :
\begin{itemize}
\item As a $\mathbb{C}[[h]]$-module, $U_h(\mathfrak{g})=U(\mathfrak{g})[[h]]$\\
\item As a Hopf algebra we have $U_h(\mathfrak{g})/hU_h(\mathfrak{g})=U_h(\mathfrak{g})$\\
\item There exists coproduct, counit and antipode that makes $U_h(\mathfrak{g})$ into a topological Hopf algebra\\
\item Any finite dimensional $\mathfrak{g}$ module $V$ extends uniquely to a unique $U_h(\mathfrak{g})$-module $V_h=V[[h]$
\end{itemize}
The indeterminate $h$ play the role of the planck constant in quantum mechanics, that's why we talk about "quantization". The category $\mathbb{C}_h$ whose objects are $U_h(\mathfrak{g})$-modules is actually a Ribbon category. This fact and the previous theorem allows us to build a functor $F_{\mathfrak{g},V}$ from the category of colored tangle to the category of $U_h(\mathfrak{g})$-modules that preserves tensor product. So to every oriented link $K$ one defines an isotopy invariant $F_{\mathfrak{g},V}(K)$ such that : $$F_{\mathfrak{g},V}(K)=\sum_{m\geq 0}F_{\mathfrak{g},V,m}(K)h^m$$

\bibliography{biblio}{}
\bibliographystyle{plain}
\end{document}
